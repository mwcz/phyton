\documentclass[12pt]{article}

\newcommand{\piRsquare}{\pi r^2}        % This is my own macro !!!

\title{My Sample \LaTeX{} Document}         % used by \maketitle
\author{\L\"{a}rs  Schl{\oe}ff\d{o}ng\"{e}n, }      % used by \maketitle
\date{July 14, 2005}                    % used by \maketitle

\begin{document}
\maketitle                      % automatic title!

I typed this file with a plain text editor.
(I used \textbf{pico} and \textbf{emacs}.)
End
    of
        paragraph.

This is my second paragraph.
The area of a circle is $\pi r^2$; again, that is $\piRsquare$.
My score on the last exam\footnote{May 23} was $95 \pm 5$.

\section{Formulae; inline vs. displayed}

I insert an inline formula by surrounding it with a pair of
single \$ symbols;  what is $x = 3 \times 5$?
For a \emph{displayed} formula, use double-\$
before and after --- include no blank lines!
$$\mu^{\alpha+3} + (\alpha^{\beta}+\theta_{\gamma}+\delta+\zeta)$$

\subsection{Numbered formulae}

Use the \emph{equation} environment to get numbered formulae, e.g.,
\begin{equation}
    y_{i+1} = x_{i}^{2n} - \sqrt{5}x_{i-1}^{n} + \sqrt{x_{i-2}^7} -1
\end{equation}

\begin{equation}
    \frac{\partial u}{\partial t} + \nabla^{4}u + \nabla^{2}u +
        \frac12    |\nabla u|^{2}~ =~ c^2
\end{equation}

\section{Acknowledgments}

Thanks to my buddies {\AE}schyulus and Chlo\"{e},
who helped me define the macro \verb9\piRsquare9
which is $\piRsquare$.
The end.

\end{document}             % End of document.

